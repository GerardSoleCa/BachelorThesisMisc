\chapter{Tools}\label{C:Tools}

\section{Source control tools}\label{S:Tools-control}
\subsection{Git}\label{SS:Tools-Git}
\subsection{GitHub}\label{SS:Tools-GitHub}
\subsection{Git source control provider extension}\label{SS:Tools-GitSource}

\section{Package management system}\label{S:Tools-pms}

\subsection{NuGet}\label{SS:Tools-NuGet}

NuGet is a free, open source developer focused package management system for the .NET platform intent on simplifying the process of incorporating third party libraries into a .NET application during development.

\subsection{Packages}\label{SS:Tools-packages}

\subsubsection{Log4net}\label{SSS:Tools-log4net}

Log4net, a port of the popular Java library log4j, is an open source library that allows .NET applications to log output to a variety of sources (e.g., console, files or SMTP). The information is logged via one or more loggers which provide a the following five logging levels: 

\begin{itemize}
\item \textbf{Debug}   
\item \textbf{Information} 
\item \textbf{Warnings}
\item \textbf{Error} 
\item \textbf{Fatal} 
\end{itemize}

\subsubsection{NUnit}\label{SSS:Tools-Nunit}

NUnit, a port from JUnit, is a unit-testing framework for all .NET languages. It is written entirely in C\# and has been completely redesigned to take advantage of many .NET language features, for example custom attributes and other reflection related capabilities. 

NUnit does not support Visual Studio integration. Instead of this it provides an external program compiled either as a console app or a GUI. This program is able to runs and execute the unit tests from an assembly.

\begin{figure}[H]\begin{center}
 \centering
  \captionsetup{justification=centering}
  %\includegraphics[scale=0.45]{pictures/appendices/tools/NuGetGUI}
  \caption{MAREA unit tests executed by NUnit GUI application\label{fig:tools-NUnitGUI}}
\end{center}\end{figure}

This framework has been especially used to test encoder layer functionalities. The listing \ref{lst:NUnit-test} shows an example of a unit test to serialize and deserialize a double with two diferent pair of parameters.

\begin{lstlisting}[language=CSharp, caption={MAREA encoder layer unit test: serialization and deserialization of a double}, classoffset=2,morekeywords={SetUp,TestCase,PerformanceTimer,ResultsManager,Description,AdaptedMareaCoder,CoderTestsConstants,Results,Console,Assert},label={lst:NUnit-test}]
 private byte[] seralizedData = null;
 private long start, serializeTicks, deserializeTicks;
 private long  clock_freq = PerformanceTimer.Clock_freq();
  
 [SetUp]
 public void RunAfterAnyTest()
 {
  serializeTicks = 0;
  deserializeTicks = 0;
 }
  
 [TestCase(0.100000234523, 0), NUnit.Framework.Description("Coder(double, System.Double)")]
 [TestCase(double.MaxValue,0)]
 public void TestDoubleM2(double oDouble, double rDouble)
 {
  for (int i = 0; i < CoderTestsConstants.CODIFICATIONS; i++)
  {
   start = PerformanceTimer.Ticks();
   seralizedData = AdaptedMareaCoder.Send(oDouble);
   serializeTicks += PerformanceTimer.TicksDifference(start);

   start = PerformanceTimer.Ticks();
   rDouble = (double)AdaptedMareaCoder.Receive(seralizedData);
   deserializeTicks += PerformanceTimer.TicksDifference(start);
  } 

  Console.WriteLine(CoderTestsConstants.MAREA2);
  Results results = ResultsManager.GetResults(serializeTicks, deserializeTicks, clock_freq, 				
  CoderTestsConstants.CODIFICATIONS, seralizedData.Length, rDouble.GetType().FullName);
	
  if (oDouble == rDouble)
  {
   Assert.True(true);
   Console.WriteLine(CoderTestsConstants.OK_STATE);
   Console.WriteLine(results.ToString());
  }
  else
  {
   Console.WriteLine(CoderTestsConstants.KO_STATE);
   Assert.True(false);
  }
 }
\end{lstlisting}

\subsubsection{Nuget Server}\label{SSS:Tools-NuGet-server}

\section{Cross platform tools}\label{S:Tools-control}
\subsection{Mono}\label{SS:Tools-Mono}
\subsection{Alter Native}\label{SS:Tools-Alternative}








