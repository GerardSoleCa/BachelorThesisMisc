\chapter{Conclusions}\label{C:Conclusions}
\section{Project Conclusions}\label{S:Project-Conclusions}
An implementation of Micro Framework has been done with the necessary features to make the existing software created for that platform work on it (HomeSense and the RFID card reader for example). Although it works well, the performance is not as good as the original, probably because the Micro Framework runs on dedicated hardware and all the consuming operations are implemented internally on the virtual machine in C. So this is probably one of the reasons why HomeSense works better on the Netduino than on the Raspberry Pi although the second one is much more powerful.
\\
\\
Another goals was to create a portable implementation. IOSharp is implemented using C\# with some calls to Linux SYSFS or Kernel functions. By using this way the implementation is architecture independent, so it is capable of running either on a standard desktop or on an ARMv6, like the Raspberry Pi. The only requirement is having an underlying Linux with the Mono interpreter executing .NET code.
\\
\\
In order to try to improve the IOSharp performance, it was translated to C++ using AlterNative which is a code translating tool. Currently, it can partially transform the original code to C++ but it still needs some work on its core because many classes are not implemented yet. The parts that AlterNative could not translate where fixed by hand, however this the amount of time saved with this tool is bigger than the required time to implement the same code from scratch. After all, once IOSharp is translated to C++, the performance is increased as it is shown in the chapter \ref{C:Performance-Test}. For example, in 10k iterations in the state of an Output GPIO the C\# version needs around 10 seconds to finish the task while the translated one only takes around 3 seconds. This implies an increase of a 62\% in performance.
\\
\\
And finally, the best of AlterNative is that the generated code is still cross platform because it uses standard C++ or libraries which are also available on different platforms, so once the source is generated it can be taken to an ARM Linux and then compile it to an executable assembly.

\subsection{Achieved Objectives}\label{Conclusions-objectives}
On Chapter \ref{C:project-overview} (Project Overview) the objectives for this thesis where listed, above are listed the results in terms of objective achievements.
\begin{itemize}
	\item \textbf{IOSharp:} all objectives had been achieved.
		\begin{itemize}
		\item \textbf{GPIO}
		\item \textbf{Interrupts}
		\item \textbf{SPI}
		\item \textbf{UART}
		\item \textbf{HomeSense}
		\end{itemize}
	\item \textbf{AlterNative:} most of the objectives where achieved.
		\begin{itemize}
		\item \textbf{Cross-Platform:} now is possible to use this tool in any operating system capable of run .NET programs.
		\item \textbf{Library:} the necessary methods for IOSharp translation are done.
		\item \textbf{Tests:} each implemented method has its own functional test.
		\item \textbf{Performance analysis:} shown in chapter \ref{C:Performance-Test}.
		\item \textbf{Translate HomeSense:} although AlterNative translates IOSharp, HomeSense is not completely translated.
		\end{itemize}
\end{itemize}

\section{Personal Conclusions}\label{S:Personal-Conclusions}
Beyond academic achievements, all the process involving this bachelor's thesis has been rewarding. This project has given me a real chance to start working with embedded strategies, protocols and it also has provided an introduction to C++ development.
\\
\\
This project has allowed me to get familiar with Micro Framework and Linux development, which represents the new paths for the embedded systems as it has been explained on the introduction of this thesis. Then, on the AlterNative part of the project, an introduction to AST patterns was done to achieve the capability to transform a representational AST in one language into another one. Also I was introduced to C++ development focused on multi-platform environments.
\\
\\
There are many skills acquired or consolidated during this time: from the initial touchdown on Micro Framework, platform invocation services (cross-language calls), delegation patterns, guidelines for application performance, etc.
\\
\\
I also have realized that the development of a project is a quite complex task and requires hard effort and dedication, but most of all a strict control of timings in order to accomplish with the established work plan.

\section{Future Work}\label{S:Future-Work}
The results of this bachelor's thesis point to several interesting directions for future work.
In case of IOSharp:
\begin{itemize}
\item \textbf{Addition of new protocols:} Currently IOSharp offers the simple GPIOs, UART and SPI but there are other common protocols or interfaces that could be developed to extend the features, such as the \gls{I2C} bus protocol. It could also be nice to implement the analogical ports or even \gls{PWM} control in I/O ports.

\item \textbf{Performance optimization:} The implementation in .NET is too much slow running on Mono, probably related to the use of the \gls{SYSFS}. It could be interesting to do some performance tests using the \gls{SYSFS} and the Kernel functions provided by Linux. Apart from this, the interrupts are not implemented using \gls{IRQ} so using \gls{IRQ} instead of polling interrupts could be a good improvement on performance although it could affect on the portability between boards, some distributions do not accept IRQ interrupts from the GPIOs.
\end{itemize}

In the case of AlterNative translating tool:
\begin{itemize}
\item \textbf{Garbage Collector:} Although AlterNative has a garbage collector implemented using the Boehm GC library, it does not get called periodically so programs with a big footprint in RAM can get out of memory easily. To solve this issue the Boehm GC could be called to release unused RAM and make it available again to the program or the system.

\item \textbf{Continuous Integration:} Changes on the core of AlterNative are very susceptible to break some implemented functionalities, this is why regression tests were created to test if the different functionalities are working well. The problem is that this test takes a huge amount of time to finish so introducing continuous integration tools such as Jenkins or Hudson can provide an easy way to quickly test these changes on a server capable of detecting new commits to a git repository. After executing all the test an inform could be mailed to know the results of it.

\item \textbf{Extend capabilities:} It could be interesting to make IOSharp work with MAREA2 which is a Middleware for distributed embedded systems in different areas such as: telecommunications, avionics, health-care, automotive, defense, etc. MAREA is a software specifically designed to fulfil Unmanned Aircraft Systems (UAS) communications and their application to the design of complex distributed UAS avionics.
\end{itemize}

\section{Environmental Impact}\label{S:Environmental-Impact}
At last but not least it is relevant to talk about the environmental impact of the work described in this document. It can be seen from the present document, this project consists in the design and development of a software application. This has not a direct environmental benefit, but IOSharp is an implementation on a high-level basis of .NET Micro Framework which is an operating system for embedded devices, so it makes easy to develop applications which helps control different kind of situations like home installations (i.e. lights, temperature or humidity) to applications capable of detect and analyse different parameters from the environment (i.e. weather and wind stations).
\\
AlterNative also reduces the power consumption required to execute a program, by removing the .NET virtual machine in which non-translated programs relay on. The translated binary runs directly on the operating system avoiding all the overhead that a virtual machine implies.