\chapter{State of the art}\label{C:State-Art}

This chapter sketches out briefly the state of the art of the embedded operating systems and its capabilities. Then according to this thesis it will be explained what is the current operating system running on bottom of HomeSense and finally why has been chosen the RaspberryPi as the target device.

\section{Embedded Systems}\label{S:Embedded-Systems}
Embedded Systems now a days are taking relevance again with the Internet of the Things, environment sensing, Wireless Sensor Networks and all new coming technologies that require low power consumption, small size, mobility environments, ...

In Embedded Systems or Resource Constrained Systems it is interesting to take a look into the Hardware platform and its capabilities, the differences between platforms, and also which tools or unique features offers to developers.

An operating system (OS) offers an interface with the hardware to make it independent from the applications that the device runs, making easy the interactions between hardware and the programs running on the machine.

An OS is an important program that makes easy to develop applications, but it is important to maintain the features that the processor offers, avoiding performance or capabilities degradation. 
This bachelor thesis is focused on constrained-resource devices, where the processing capabilities and memory resources are limited, is fundamental to respect the above criteria.

\subsection{Operating Systems Architectures}\label{Operating-Systems-Architectures}
In general, there are three types of operating system architectures for embedded devices, which are based on how applications are executed or included into the OS.

\begin{itemize}
\item \textbf{Monolithic:} The OS and the applications are combined into a single program. Normally in this situations the embedded device runs in the same process the OS and the program written to it. This type of architecture makes difficult to include new functions without rewriting much of the code.

\item \textbf{Modular:} The OS is running as a standalone program in the processor and has de ability to load programs to it self as modules. In terms of the development, it's possible to develop applications without writing in the core of the OS. Normally using modules developers can expand the capabilities of its software.

\item \textbf{Virtual-Machine:} The OS creates an abstraction layer of its underlying hardware, this abstracted layer is common in every device that implements that virtual-machine. Using this type of operating system provides a helpful tool to achieve the well known slogan \textit{write once, run anywhere}. Although using virtual-machine devices simplifies the development on multiple devices, the performance of the platform normally will be reduced and in Real Time environments it isn't recommended to use it.
\end{itemize}

\subsection{Embedded Operating Systems}\label{Embedded-Operating-Systems}
There is a wide range of Embedded Operating Systems each of them has strengths and weaknesses, below different OS are described and compared.
 
\begin{itemize}
\item \textbf{TinyOS} is a popular open source OS for wireless constrained devices, many of them used in wireless sensor networks. It provides software abstractions from the underlying hardware. It is focused on wireless communications offering stacks for 802.15.4 and ZigBee. It also supports secure networking and implements a RPL taking in mind the forthcoming routing protocol for low power and lossy networks.
However, TinyOS changes how programs should be developed, it intended to use non-blocking programming which means that it isn't prepared for long processing functions. For example, when TinyOS called to send a message the function will return immediately and after a while the send will be processed and after then, TinyOS will make a callback to a function, for example send()'s callback will be sendDone().

\item \textbf{FreeRTOS} is a free real-time OS that supports over 34 architectures and it is being developed by professionals under strict quality controls and robustness. Is used from toys to aircraft navigation and it is interesting for its real-time qualities. It has a very small memory footprint (RAM usage) and very fast execution, based on hard real-time interruptions performed by queues and semaphores. Apart from this, there are not constraints on the maximum number of tasks neither the priority levels that can be used on tasks. 

\item \textbf{Contiki:} asdf asdf asdf.

\item \textbf{Micro Framework .NET:} asdf asdf asdf.
\end{itemize}



\section{Micro Framework .NET}\label{S:MicroFramework}


\subsection{NETMF enabled devices}\label{SS:MicroFramework-Devices}

MicroFramework can run on CLR enabled devices that are MicroFramework compilant with it's specifications. In this bachelor thesis a Netduino Plus (from XXXXXLabs) has been used to test, understand and code sample code in order to know how Microframework works. Apart from this Netduino there are other devices like the cerbuino,which are also capable to run CLR code.

Among this, there are more powerful devices that can run and execute simple graphics programs.


\subsection{Software Development Kit}\label{SS:MicroFramework-SDK}
As Micro Framework is similar to an operating system it has


\subsection{Visual Studio}\label{SS:MicroFramework-IDE}

\section{RaspberryPi and MicroFramework}\label{S:RaspberryPi-NETMF}
Before starting with the development a search was done in order to know if there was any project involving the port of the MicroFramework to Linux devices using, for example, Mono. Nothing was found.
There is currently one implementation of MicroFramework for Linux, but it only works in a resource-constrained device called Edy Linux.

RaspberryPi has many implementations in different languages involving its IO ports, many are written in C and Python, others are for example in Java. But when speaking in terms of .NET/C\# there is an important lack in IO implementations, below are exposed the most important ones that where found.

\begin{itemize}
\item \textbf{BlaBlaBla:}
\item \textbf{BlaBlaBla:}
\end{itemize}

Although the XXX library is really interesting according to it's description of functionality, it doesn't 